%
% -------------------- LaTeX-Template für Abschlussarbeiten --------------------
%
% Autor:    Simon Lehmann <simon.lehmann@gmx.de>
% Version:  2.0 (2009-09-10)
% 
% Versionshistorie: 
% | 2.0 | 2009-09-10 | Umbenennung FH Wiesbaden in Hochschule RheinMain
% | 1.1 | 2008-11-15 | Layout verbessert
% | 1.0 | 2008-06-19 | Urversion
% 
% Das Template verwendet die KOMA-Script Dokumentenklassen und setzt einige
% LaTeX Pakete voraus, die unter Umständen nicht im Standardumfang der lokalen
% LaTeX Distribution enthalten sind. Erhältlich sind diese Pakete über das CTAN
% (http://www.ctan.org/). Zur Installation siehe die Dokumentation des jeweili-
% gen Pakets und der LaTeX Distribution.
%
% Weiterhin ist das Template auf die Verwendung von UTF-8 als Zeichenkodierung
% ausgelegt. Dadurch können Umlaute und andere Sonderzeichen direkt verwendet
% werden.
%
% Es sind für die variablen Felder auf der Titelseite und den formalen Erklär-
% ungen Befehle definiert, mit denen diese gesetzt werden können. Siehe dazu
% weiter unten den Abschnitt 'Vorspann'.
% 
% Für die Glossarerstellung wird das glossaries Paket als Nachfolger von
% glossary verwendet. Das Paket ist leider noch nicht in den meisten Distribu-
% tionen von Haus aus verfügbar und muss eventuell manuell über das CTAN (s.o.)
% bezogen werden. Doku siehe <http://tug.ctan.org/tex-archive/macros/latex/
% contrib/glossaries/glossaries-manual.html>
%
%
% Zum Übersetzen sind in der Regel folgende Durchläufe nötig:
%  1. latex arbeit
%  2. bibtex arbeit (nur wenn schon Zitate im Text vorhanden sind)
%  3. makeglossaries arbeit
%  4. latex arbeit
%  5. latex arbeit (damit Referenzen und Verzeichnisse stimmen)
%
% --------------------------- Dokumenteneinrichtung ----------------------------
% Einsteigern wird empfohlen direkt zum Abschnitt 'Vorspann' weiter unten zu
% springen und die dokumentenweiten Einstellungen zunächst nicht zu verändern.
%
\documentclass[ngerman,a4paper,11pt,twoside,openright,cleardoubleempty,halfparskip]{scrreprt}

% Erforderliche Pakete
\usepackage[utf8]{inputenc}
\usepackage{babel}
\usepackage{graphicx}
\usepackage{fancyhdr}
\usepackage{multirow}
\usepackage{ifthen}
\usepackage{calc}
\usepackage{tabularx}

% Laden von weiteren, nützlichen Paketen (je nach Bedarf)
%\usepackage{subfigure} 
%\usepackage{varioref}
%\usepackage{framed}
%\usepackage{longtable}
%\usepackage{floatflt}
%\usepackage{amsmath}
%\usepackage{url}
% ...

% Pakete für Verzeichnisse. Diese sind hier aufgeführt, da mindestens das
% 'glossaries' Paket nach z.B. hyperref geladen werden muss.
\usepackage{bibgerm}
\usepackage[toc,acronym]{glossaries}

% Glossar(e) laden und erstellen
\makeglossaries
\loadglsentries{glossar} % Name der Glossardatei (ohne .tex)

% Befehlsdefinitionen
\input{commands}

% Globale Formatierungseinstellungen
\renewcommand{\encodingdefault}{OT1}
\renewcommand{\familydefault}{cmss} % Schriftfamilie auf Sans Serif
\renewcommand{\glsdisplayfirst}[4]{\textit{#1#4}}
\setcapindent{1em}

% Anpassung der Ränder und Breitenverhältnisse
% Der letzte Wert in oddsidemargin bestimmt die Bindekorrektur
\setlength{\oddsidemargin}{2cm - 1in + 0.5cm}
\setlength{\textwidth}{\paperwidth - (1in + \hoffset) - \oddsidemargin - 4cm}
\setlength{\evensidemargin}{\paperwidth - (1in + \hoffset)*2 - \oddsidemargin - \textwidth}
\setlength{\marginparwidth}{4cm - \marginparsep{} - 1cm}

% Wenn größerer Zeilenabstand gewünscht ist, je nach Schriftgröße:
% Abstand        10pt    11pt    12pt
% -----------------------------------
% anderthalb     1.25    1.21    1.24
% doppelt        1.67    1.62    1.66
%
%\renewcommand{\baselinestretch}{1.21}

% Kopf- und Fußzeilen einrichten
\pagestyle{fancyplain}
\setlength{\headwidth}{\textwidth}
\addtolength{\headwidth}{\marginparsep}
\addtolength{\headwidth}{\marginparwidth}
\renewcommand{\chaptermark}[1]{\markboth{#1}{}}
\renewcommand{\sectionmark}[1]{\markright{\thesection\ #1}}
\lhead[\fancyplain{}{\bfseries\thepage}]%
	{\fancyplain{}{\bfseries\rightmark}}
\rhead[\fancyplain{}{\bfseries\leftmark}]%
	{\fancyplain{}{\bfseries\thepage}}
\cfoot{}

\begin{document}
% ---------------------------------- Vorspann ----------------------------------
% In den folgenden Zeilen werden die wichtigsten Informationen zur Arbeit ge-
% setzt, die dann im Dokument an den entsprechenden Stellen eingefügt werden.
%
\title{Titel der Arbeit}
\author{Name des Autoren}
\supervisor{Name des Referenten}
\cosupervisor{Name des Korreferenten}
\date{Datum der Abgabe}
\city{Ort}

% Folgende Befehle steuern, welchen Verbreitungsformen zugestimmt wird. Nicht
% gewünschte Verbreitungsformen entsprechend auskommentieren.
%
\publishlibrary  % Einstellung der Arbeit in die Bibliothek der FHW
\publishtitle    % Veröffentlichung des Titels der Arbeit im Internet
\publishdocument % Veröffentlichung der Arbeit im Internet

% Ab hier sollten keine Änderungen des Vorspanns nötig sein!
% -> Weiter zum Hauptteil

\pagenumbering{roman} % Bis zum ersten Kapitel mit römischen Seitenzahlen

\input{titel} % Titelseite (wird automatisch mit Werten von oben gefüllt)
\clearemptydoublepage % Sorgt dafür, dass die nächste Seite rechts beginnt

\include{vorspann} % Erklärung zur Prüfungsordnung und Verwendung der Arbeit
\clearemptydoublepage

\tableofcontents % Inhaltsverzeichnis generieren

\clearemptydoublepage
\pagenumbering{arabic}

% --------------------------------- Hauptteil ----------------------------------
% Im diesen Teil werden die einzelnen Kapitel eingefügt, die sinnvollerweise
% im Verzeichnis 'kapitel' abgelegt werden.

\chapter{Einleitung}

Hier Beginnt das Dokument \ldots

%\include{kapitel/zweileitung}

% und so weiter ...

% ---------------------------------- Anhänge -----------------------------------
% In diesem Teil werden alle Anhänge eingefügt, die auch als ganz normale Kapi-
% tel abgelegt werden.
\appendix

%\include{kapitel/einanhang}

% Ausgabe des Glossars (oder der Glossare, wenn mehrere definiert sind)
\printglossaries

% Ausgabe des Literaturverzeichnisses
\bibliographystyle{geralpha}
\bibliography{literatur}

\end{document}
